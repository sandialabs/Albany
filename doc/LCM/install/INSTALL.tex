\documentclass[10pt,a4paper]{article} \usepackage[utf8]{inputenc}
\usepackage{amsmath} \usepackage{amsfonts} \usepackage{amssymb}
\usepackage{graphicx} \usepackage{fancyvrb} \usepackage{color}
\usepackage[left=2cm,right=2cm,top=2cm,bottom=2cm]{geometry}

\let\oldv\verbatim \let\oldendv\endverbatim

\def\verbatim
{\par\setbox0\vbox\bgroup\oldv}

\def\endverbatim
{\oldendv\egroup\fboxsep0pt \noindent\colorbox[gray]{0.95}{\usebox0}\par}

\author{The Albany/LCM team\thanks{Original version by Juli\'an R\'imoli}}
\title{Instructions for Installing Albany/LCM and Trilinos on Fedora 20}
\begin{document}

\maketitle

\section{Introduction}
This document describes the necessary steps to install Albany/LCM and
Trilinos on a machine with Fedora Linux. The procedures described
herein were tested using Fedora 20.

\section{Required packages}
After installing Fedora, the following packages should be installed
using the {\tt yum} command:
\begin{verbatim}
blas
blas-devel
lapack
lapack-devel
openmpi
openmpi-devel
netcdf
netcdf-devel
netcdf-static
netcdf-openmpi
netcdf-openmpi-devel
hdf5
hdf5-devel
hdf5-static
hdf5-openmpi
hdf5-openmpi-devel
boost
boost-devel
boost-static
boost-openmpi
boost-openmpi-devel
cmake
gcc-c++
git
\end{verbatim}

For example, to install the first package you should type:
\begin{verbatim}
sudo yum install blas
\end{verbatim}

Make sure that all these packages are installed, specially of you
create a script to do so. If a package is not installed because of a
typo then the compilation will fail.

Optional but strongly recommended packages:
\begin{verbatim}
clang
clang-devel
gitk
\end{verbatim}

\section{Git repository setup with Github}

In a web browser go to \verb+www.github.com+, create an account and
set up ssh public keys. If you require push privilges for Albany,
email Glen Hansen at \verb+gahanse@sandia.gov+ and let him know
that. On the other hand, if you require push privileges for Trilinos,
it is best if you contact the Trilinos developers directly. Go to
\verb+www.trilinos.org+ for more information.

It is strongly recommended that you join the \verb+AlbanyLCM+ Google
group to receive commit notices. Go to
\verb+groups.google.com/forum/#!forum/albanylcm+ and join the
group. You can also browse the source code at
\verb+github.com/gahansen/Albany+.

\section{Directory structure}
In your home directory, create a directory with the name \verb+LCM+:
\begin{verbatim}
mkdir LCM
\end{verbatim}

Change directory to the newly created one:
\begin{verbatim}
cd LCM
\end{verbatim}

Now, check out the latest version of Trilinos:
\begin{verbatim}
git clone git@github.com:nschloe/trilinos.git Trilinos
\end{verbatim}

Finally, check out the latest version of Albany:
\begin{verbatim}
git clone https://github.com/gahansen/Albany.git Albany
\end{verbatim}

At this point, the directory structure should look like this:
\begin{verbatim}
LCM
  |- Albany
  |- Trilinos
\end{verbatim}

\section{Installation scripts}
Copy the installation scripts inside the directory
\verb+LCM/Albany/doc/LCM/build+ to the top-level \verb+LCM+
directory. The necessary scripts are:
\begin{verbatim}
albany-config.sh
build-all.sh
build.sh
env-all.sh
env-single.sh
trilinos-config.sh
\end{verbatim}

Once copied, go to the top-level \verb+LCM+ directory, open the
\verb+env-all.sh+ and \verb+env-single.sh+ scripts and make sure they
match your environment. If you do not want to tinker with any of this,
just change the email addresses for tests reports at the end of
\verb+env-single.sh+ and make sure all the scripts are executable and read
only:
\begin{verbatim}
cd ~/LCM
chmod +x *.sh
chmod -r *.sh
\end{verbatim}

The \verb+build.sh+ and \verb+build-all.sh+ scripts perform different
actions according to the name with which they are invoked. See
\verb+LCM/Albany/doc/LCM/build/README+ for more details. 

For example, the following symbolic links will create separate
commands for clean up, configuring and testing:
\begin{verbatim}
ln -s build.sh clean.sh
ln -s build.sh config.sh
ln -s build.sh test.sh
\end{verbatim}
They could also be combined for convenience:
\begin{verbatim}
ln -s build.sh clean-config.sh
ln -s build.sh clean-config-build.sh
ln -s build.sh clean-config-build-test.sh
ln -s build.sh config-build.sh
ln -s build.sh config-build-test.sh
\end{verbatim}

\section{Environment variables}
The following additions to the command shell environment variables are
required for proper Albany compilation and execution in this setup:
\begin{verbatim}
export PATH=/usr/lib64/openmpi/bin:$PATH
export LD_LIBRARY_PATH=/usr/lib64/openmpi/lib:$LD_LIBRARY_PATH
\end{verbatim}

\section{Configuring and compiling}
First, configure and compile Trilinos. Within the top-level \verb+LCM+
directory type:
\begin{verbatim}
./config-build.sh trilinos [toolchain] [build_type] [#_processors]
\end{verbatim}

For example, if you want to use the \verb+GCC+ toolchain to build a
release version of the code using 16 processors, type:
\begin{verbatim}
./config-build.sh trilinos gcc release 16
\end{verbatim}

Finally, repeat the procedure for Albany:
\begin{verbatim}
./config-build.sh albany [toolchain] [build_type] [#_processors]
\end{verbatim}

For example, if you want to use the \verb+GCC+ toolchain to build a
release version of the code using 16 processors, type:
\begin{verbatim}
./config-build.sh albany gcc release 16
\end{verbatim}

Note that to compile a version of Albany with a specific toolchain and
build type, the corresponding version of Trilinos must exist.

Currently the options for the toolchain are \verb+gcc+ and
\verb+clang+, and for build type are \verb+debug+ and
\verb+release+. The \verb+clang+ toolchain requires installation of
the \verb+clang+ and \verb+clang-devel+ packages.

\section{After initial setup}
The procedure described before configures and compiles the code. From
now on, configuration is no longer required so you can rebuild the
code after any modification by simply using the \verb+build.sh+
script. For example:
\begin{verbatim}
./build.sh albany gcc release 16
\end{verbatim}

There are times when it is necessary to reconfigure, for example when
adding or deleting files under the \verb+LCM/Albany/src/LCM+
directory. This is generally anounced in the commit notices.

Also, note that both Trilinos and Albany are heavily templetized C++
codes. Building the debug version of Albany requires large amounts of
memory because the huge size of the symbolic information required for
debugging. Thus, if the compiling procedure stalls, try reducing the
number of processors.

\section{Policies}
Albany is a simulation code for researchers by researchers. As such,
vibrant development of new and exciting capabilities is strongly
encouraged. For these reasons, don't be afraid to commit changes to
the master git repository. We only ask that you don't break
compilation or testing. So please make sure that the tests pass before
you commit changes.

In addition, within LCM we strongly encourage you to follow the C++
Google style guide that can be found at
\verb+http://google-styleguide.googlecode.com/svn/trunk/cppguide.html+. This
style is somewhat different to what is currently used in the rest of
Albany, but we believe that the Google style helps the developer more
in that it advocates more style differentiation between the different
syntactic elements of C++. This in turn makes reading code easier and
helps to avoid coding errors.

\end{document}