\documentclass[10pt,a4paper]{article} \usepackage[utf8]{inputenc}
\usepackage{amsmath} \usepackage{amsfonts} \usepackage{amssymb}
\usepackage{graphicx} \usepackage{fancyvrb} \usepackage{color}
\usepackage[left=2cm,right=2cm,top=2cm,bottom=2cm]{geometry}

\let\oldv\verbatim \let\oldendv\endverbatim

\def\verbatim
{\par\setbox0\vbox\bgroup\oldv}

\def\endverbatim
{\oldendv\egroup\fboxsep0pt \noindent\colorbox[gray]{0.95}{\usebox0}\par}

\author{The Albany/LCM team\thanks{Original version by Juli\'an R\'imoli}}
\title{Instructions for Installing Albany/LCM and Trilinos on Fedora 21}
\begin{document}

\maketitle

\section{Introduction}
This document describes the necessary steps to install Albany/LCM and
Trilinos on a machine with Fedora Linux. The procedures described
herein were tested using Fedora 22 and Fedora 21.

If you want a shortcut obtain the script \verb+install_albany.sh+
(stored in \verb+Albany/doc/LCM/build+) and then try:
\begin{verbatim}
./install_albany
\end{verbatim}
This will install and build Trilinos and Albany in the current
directory.  If the script does not complete it will tell you why and
with help from this document you will be able to complete the install.

\section{Required packages}
After installing Fedora, the following packages should be installed
using the {\tt dnf} command in Fedora 22 or the {\tt yum} command in
Fedora 21:
\begin{verbatim}
blas
blas-devel
lapack
lapack-devel
openmpi
openmpi-devel
netcdf
netcdf-devel
netcdf-static
netcdf-openmpi
netcdf-openmpi-devel
hdf5
hdf5-devel
hdf5-static
hdf5-openmpi
hdf5-openmpi-devel
boost
boost-devel
boost-static
boost-openmpi
boost-openmpi-devel
matio
matio-devel
cmake
gcc-c++
git
hwloc-libs
hwloc-devel
\end{verbatim}

For example, to install the first package you should type in Fedora 22
\begin{verbatim}
sudo dnf install blas
\end{verbatim}
or in Fedora 21
\begin{verbatim}
sudo yum install blas
\end{verbatim}

Make sure that all these packages are installed, specially if you
create a script to do so. If a package is not installed because of a
typo then the compilation will fail.

Optional but strongly recommended packages:
\begin{verbatim}
clang
clang-devel
gitk
\end{verbatim}

\section{Git repository setup with Github}

In a web browser go to \verb+www.github.com+, create an account and
set up ssh public keys. If you require push privilges for Albany,
email Glen Hansen at \verb+gahanse@sandia.gov+ and let him know
that. On the other hand, if you require push privileges for Trilinos,
it is best if you contact the Trilinos developers directly. Go to
\verb+www.trilinos.org+ for more information.

It is strongly recommended that you join the \verb+AlbanyLCM+ Google
group to receive commit notices. Go to
\verb+groups.google.com/forum/#!forum/albanylcm+ and join the
group. You can also browse the source code at
\verb+github.com/gahansen/Albany+.

\section{Directory structure}
In your home directory, create a directory with the name \verb+LCM+:
\begin{verbatim}
mkdir LCM
\end{verbatim}

Change directory to the newly created one:
\begin{verbatim}
cd LCM
\end{verbatim}

Now, check out the latest version of Trilinos. If you have an account on
\verb+software.sandia.gov+, then use:
\begin{verbatim}
git clone software.sandia.gov:/space/git/Trilinos Trilinos
\end{verbatim}
Otherwise clone with:
\begin{verbatim}
git clone git@github.com:trilinos/trilinos.git Trilinos
\end{verbatim}
This last public Trilinos repository is a few days behind the one at
\verb+software.sandia.gov+ which for most purposes will not be an issue.

Finally, check out the latest version of Albany:
\begin{verbatim}
git clone git@github.com:gahansen/Albany.git Albany
\end{verbatim}

At this point, the directory structure should look like this:
\begin{verbatim}
LCM
  |- Albany
  |- Trilinos
\end{verbatim}

\section{Installation scripts}
Create symbolic links to the installation scripts inside the directory
\verb+LCM/Albany/doc/LCM/build+ to the top-level \verb+LCM+
directory. If you intend to modify these scripts, it is better to copy
them. The necessary scripts are:
\begin{verbatim}
albany-config.sh
build-all.sh
build.sh
env-all.sh
env-single.sh
trilinos-config.sh
\end{verbatim}

Once this is done, go to the top-level \verb+LCM+ directory, open the
\verb+env-all.sh+ and \verb+env-single.sh+ scripts and make sure they
match your environment. If you do not want to tinker with any of this,
just change the email addresses for tests reports at the end of
\verb+env-single.sh+ and make sure all the scripts are executable and
read only:
\begin{verbatim}
cd ~/LCM
chmod +x *.sh
chmod -r *.sh
\end{verbatim}

The \verb+build.sh+ and \verb+build-all.sh+ scripts perform different
actions according to the name with which they are invoked. See
\verb+LCM/Albany/doc/LCM/build/README+ for more details. 

For example, the following symbolic links will create separate
commands for clean up, configuring and testing:
\begin{verbatim}
ln -s build.sh clean.sh
ln -s build.sh config.sh
ln -s build.sh test.sh
\end{verbatim}
They could also be combined for convenience:
\begin{verbatim}
ln -s build.sh clean-config.sh
ln -s build.sh clean-config-build.sh
ln -s build.sh clean-config-build-test.sh
ln -s build.sh config-build.sh
ln -s build.sh config-build-test.sh
\end{verbatim}
There is also a script
\verb+LCM/Albany/doc/LCM/install/albany-lcm-symlinks.sh+
that will create the appropriate symbolic links. 

\section{Environment variables}
The following additions to the command shell environment variables are
required for proper Albany compilation and execution in this setup:
\begin{verbatim}
export PATH=/usr/lib64/openmpi/bin:$PATH
export LD_LIBRARY_PATH=/usr/lib64/openmpi/lib:$LD_LIBRARY_PATH
\end{verbatim}

\section{Configuring and compiling}
First, configure and compile Trilinos. Within the top-level \verb+LCM+
directory type:
\begin{verbatim}
./config-build.sh trilinos [architecture] [toolchain] [build_type] [# processors]
\end{verbatim}

For example, if you want to use the \verb+GCC+ toolchain to build a
serial release version of the code using 16 processors, type:
\begin{verbatim}
./config-build.sh trilinos serial gcc release 16
\end{verbatim}

Finally, repeat the procedure for Albany:
\begin{verbatim}
./config-build.sh albany [architecture] [toolchain] [build_type] [# processors]
\end{verbatim}

For example, if you want to use the GCC toolchain to build a
serial release version of the code using 16 processors, type:
\begin{verbatim}
./config-build.sh albany serial gcc release 16
\end{verbatim}

Note that to compile a version of Albany with a specific architecture,
toolchain and build type, the corresponding version of Trilinos must
exist.

Note also that the architecture option refers to the thread
parallelism model that the code will use by means of the Kokkos
package in Trilinos. Currently the sopported models are: \verb+serial+
that works for all supported compilers, \verb+openmp+ that works with
the GCC and Intel toolchains, \verb+pthreads+ that works for all
supported compilers, and \verb+cuda+ that is only supported for the
GCC toolchain. The installation and configuration of the Cuda
framework is complex. Much more detailed information can be found at
\verb+http://developer.nvidia.com/cuda+.

Currently the options for the toolchain are \verb+gcc+, \verb+clang+
and \verb+intel+ if the Intel compilers are installed, and for build
type are \verb+debug+ (includes symbolic information), \verb+release+
(optimization enabled), \verb+profile+ (symbolic information and
optimization enabled for profiling) and \verb+small+ (minimizes size
of executables). The \verb+clang+ toolchain requires installation of
the \verb+clang+ and \verb+clang-devel+ packages.

\section{After initial setup}
The procedure described before configures and compiles the code. From
now on, configuration is no longer required so you can rebuild the
code after any modification by simply using the \verb+build.sh+
script. For example:
\begin{verbatim}
./build.sh albany gcc release 16
\end{verbatim}

There are times when it is necessary to reconfigure, for example when
adding or deleting files under the \verb+LCM/Albany/src/LCM+
directory. This is generally anounced in the commit notices.

Also, note that both Trilinos and Albany are heavily templetized C++
codes. Building the debug version of Albany requires large amounts of
memory because the huge size of the symbolic information required for
debugging. Thus, if the compiling procedure stalls, try reducing the
number of processors.

\section{Running and debugging LCM} 

After building Albany, you might want to run and/or debug the code. 
Tools were built in Trilinos (decomp, epu, etc.) that are necessary for parallel execution. 
Please modify your path
\begin{verbatim}
export PATH=$HOME/LCM/trilinos-install-gcc-release/bin:$PATH
\end{verbatim}
and include the necessary libraries
\begin{verbatim}
export LD_LIBRARY_PATH=$HOME/LCM/trilinos-install-gcc-release/lib:$LD_LIBRARY_PATH
\end{verbatim}
prior to execution. Please note that in this specific case, we are pointing to 
\verb+gcc-release+.
If we attempt to execute the debug version, the \verb+LD_LIBRARY_PATH+ will be invalid.
One is required to change the  \verb+LD_LIBRARY_PATH+ for each toolchain and build type.
To select toolchains and build types for execution, one can create configuration scripts in his/her LCM
directory. Each configuration script might be named something akin to \verb+LCM_gcc_release.conf+ 
and contain the necessary changes to \verb+PATH+ and \verb+LD_LIBRARY_PATH+. For example,
prior to debugging the code one would construct the configuration file  
\verb+$HOME/LCM/LCM_gcc_debug.conf+
\begin{verbatim}
export PATH=$HOME/LCM/trilinos-install-gcc-debug/bin:$PATH
export LD_LIBRARY_PATH=/$HOME/LCM/trilinos-install-gcc-debug/lib:$LD_LIBRARY_PATH
\end{verbatim}
and then \verb+source $HOME/LCM/LCM_gcc_debug.conf+ prior to execution. We note that these 
environmental variables are temporarily set by the build scripts. However, one may have a total of six
toolchains and build types. Currently, the user must specify the environmental variables 
necessary for execution. 

\section{Policies}
Albany is a simulation code for researchers by researchers. As such,
vibrant development of new and exciting capabilities is strongly
encouraged. For these reasons, don't be afraid to commit changes to
the master git repository. We only ask that you don't break
compilation or testing. So please make sure that the tests pass before
you commit changes.

In addition, within LCM we strongly encourage you to follow the C++
Google style guide that can be found at
\verb+http://google-styleguide.googlecode.com/svn/trunk/cppguide.html+.
This style is somewhat different to what is currently used in the rest
of Albany, but we believe that the Google style helps the developer
more in that it advocates more style differentiation between the
different syntactic elements of C++. This in turn makes reading code
easier and helps to avoid coding errors.

\end{document}
