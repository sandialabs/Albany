\documentclass{article}
\usepackage[utf8]{inputenc}
\usepackage{amsmath}
\usepackage{amsfonts}
\usepackage{amssymb}
\usepackage{graphicx}
\usepackage{fancyvrb}
\usepackage{color}
\usepackage[left=2cm,right=2cm,top=2cm,bottom=2cm]{geometry}

\newcommand{\trilinos}{\textsc{Trilinos}}
\newcommand{\albany}{\textsc{Albany}}
\newcommand{\lcm}{\textsc{LCM}}
\newcommand{\kokkos}{\textsc{Kokkos}}

\let\oldv\verbatim \let\oldendv\endverbatim

\def\verbatim
{\par\setbox0\vbox\bgroup\oldv}

\def\endverbatim
{\oldendv\egroup\fboxsep0pt \noindent\colorbox[gray]{0.95}{\usebox0}\par}

\author{The \albany{}/\lcm{} team\thanks{Original version by Juli\'an
    R\'imoli}} \title{Installing \albany{}/\lcm{} and
  \trilinos{} on Fedora 22/23}
\begin{document}

\maketitle

\section{Introduction}
This document describes the necessary steps to install
\albany{}/\lcm{} and \trilinos{} on a machine with Fedora Linux. The
procedures described herein were tested using Fedora 22 and Fedora 23.
If you want a shortcut obtain the script \verb+install_albany.sh+
(stored in \verb+Albany/doc/LCM/install+) and then try
\begin{verbatim}
./install_albany
\end{verbatim}
This will install and build \trilinos{} and \albany{} in the current
directory.  If the script does not complete it will tell you why and
with help from this document you will be able to complete the install.

\section{Required Packages}
The following packages should be installed using the {\tt dnf} command
\begin{verbatim}
blas
blas-devel
lapack
lapack-devel
openmpi
openmpi-devel
netcdf
netcdf-devel
netcdf-static
netcdf-openmpi
netcdf-openmpi-devel
hdf5
hdf5-devel
hdf5-static
hdf5-openmpi
hdf5-openmpi-devel
boost
boost-devel
boost-static
boost-openmpi
boost-openmpi-devel
matio
matio-devel
cmake
\end{verbatim}

\begin{verbatim}
gcc-c++
git
hwloc-libs
hwloc-devel
environment-modules
\end{verbatim}

For example, to install the first package you should type
\begin{verbatim}
sudo dnf install blas
\end{verbatim}

Make sure that all these packages are installed, specially if you
create a script to do so. If a package is not installed because of a
typo then the compilation will fail.

It may be necessary to logout and login for the module alias from the
\verb+environment-modules+ package to become active.

Optional but strongly recommended packages:
\begin{verbatim}
clang
clang-devel
gitk
\end{verbatim}

\section{Repository Setup with GitHub}

In a web browser go to \verb+www.github.com+, create an account and
set up ssh public keys. If you require push privilges for \albany{},
email Glen Hansen at \verb+gahanse@sandia.gov+ and let him know
that. On the other hand, if you require push privileges for \trilinos{},
it is best if you contact the \trilinos{} developers directly. Go to
\verb+www.trilinos.org+ for more information.

It is strongly recommended that you join the \verb+AlbanyLCM+ Google
group to receive commit notices. Go to
\verb+groups.google.com/forum/#!forum/albanylcm+ and join the
group. You can also browse the source code at
\verb+github.com/gahansen/Albany+.

\section{Directory Structure}
In your home directory, create a directory with the name \verb+LCM+:
\begin{verbatim}
mkdir LCM
\end{verbatim}

Change directory to the newly created one:
\begin{verbatim}
cd LCM
\end{verbatim}

Check out the latest version of \trilinos{}, which is hosted now on
GitHub:
\begin{verbatim}
git clone git@github.com:trilinos/Trilinos.git Trilinos
\end{verbatim}

Finally, check out the latest version of \albany{}:
\begin{verbatim}
git clone git@github.com:gahansen/Albany.git Albany
\end{verbatim}

At this point, the directory structure should look like this:
\begin{verbatim}
LCM
  |- Albany
  |- Trilinos
\end{verbatim}

\section{Environment Variables}
In \verb+~/.bashrc+, the following variables are needed:
\begin{verbatim}
export LCM_DIR=~/LCM
export MODULEPATH=$LCM_DIR/Albany/doc/LCM/modulefiles
\end{verbatim}
The \verb+LCM_DIR+ variable should contain the location of the
top-level \verb+LCM+ directory.

\section{Installation Scripts}
Create symbolic links to the installation scripts inside the directory
\verb+LCM/Albany/doc/LCM/build+ to the top-level \verb+LCM+
directory. If you intend to modify these scripts, it is better to copy
them. The necessary scripts are:
\begin{verbatim}
albany-config.sh
build-all.sh
build.sh
env-all.sh
env-single.sh
trilinos-config.sh
\end{verbatim}

Once this is done, go to the top-level \verb+LCM+ directory, open the
\verb+env-all.sh+ and \verb+env-single.sh+ scripts and make sure they
match your environment. If you do not want to tinker with any of this,
just change the email addresses for tests reports at the end of
\verb+env-single.sh+ and make sure all the scripts are executable and
read only:
\begin{verbatim}
cd ~/LCM
chmod 0555 *.sh
\end{verbatim}

The \verb+build.sh+ and \verb+build-all.sh+ scripts perform different
actions according to the name with which they are invoked. This is
accomplshed by creating symlinks to build.sh and using them to run
it. For example:
\begin{itemize}
\item clean.sh will delete all traces of the corresponding build and
  will create a new configuration script based on the corresponding
  template.

\item config.sh will attempt to reconfigure the build.

\item build.sh (original name) will build using cmake.

\item test.sh will run the cmake tests.

\item mail.sh will mail the results of the ctest to the email address
  configured in env-single.sh.

\item symlinks with combinations of the above
  (e.g. clean-config-build.sh) will perform the specified actions in
  sequence. See build.sh for valid sequences.
\end{itemize}
For example, the following symbolic links will create separate
commands for clean up, configuring and testing:
\begin{verbatim}
ln -s build.sh clean.sh
ln -s build.sh config.sh
ln -s build.sh test.sh
\end{verbatim}
They could also be combined for convenience:
\begin{verbatim}
ln -s build.sh clean-config.sh
ln -s build.sh clean-config-build.sh
ln -s build.sh clean-config-build-test.sh
ln -s build.sh config-build.sh
ln -s build.sh config-build-test.sh
\end{verbatim}
There is also a script
\verb+LCM/Albany/doc/LCM/install/albany-lcm-symlinks.sh+
that will create the appropriate symbolic links. 

It is recommended that at least the *-config.sh be made read and
execute only, as erroneous modifications to the build.sh script may
result in their being overwritten.

\section{Parallel Schwarz and DTK}
\label{sec:dtk}

This section applies only if using the alternating Schwarz coupling
method in parallel by means of the Data Transfer Kit (\verb+DTK+).
Otherwise it can be safely ignored.

The current parallel implementation of the Schwarz method requires
\verb+DTK+, which is tightly integrated to \trilinos{}, specifically
\verb+STK+. Go to the the top-level \verb+LCM+ directory and create a
symbolic link to the \verb+DTK+ CMake fragment that resides in
\verb+LCM/Albany/doc/LCM/build+, then clone the \verb+DTK+ repository
inside the \trilinos{} directory:
\begin{verbatim}
cd ~/LCM
ln -s Albany/doc/LCM/build/dtk-frag.sh .
cd Trilinos
git clone git@github.com:ORNL-CEES/DataTransferKit.git
\end{verbatim}
The configuration scripts will detect the presence of \verb+DTK+ and
configure it appropriately. Also, parallel Schwarz will be enabled when
compiling Albany.

\section{Modules}
\label{sec:modules}

Modules are used to create different environments for the
configuration and compilation of both \albany{} and \trilinos{}. To
see the available modules that correspond to different thread models,
compilers and build types:
\begin{verbatim}
module avail
\end{verbatim}
This results in something like:
\begin{verbatim}
------------------ /home/amota/LCM/Albany/doc/LCM/modulefiles ------------------
openmp-gcc-debug     pthreads-gcc-release serial-gcc-debug
openmp-gcc-profile   pthreads-gcc-small   serial-gcc-profile
openmp-gcc-release   serial-clang-debug   serial-gcc-release
openmp-gcc-small     serial-clang-profile serial-gcc-small
pthreads-gcc-debug   serial-clang-release
pthreads-gcc-profile serial-clang-small
\end{verbatim}
The naming convention for the modules follows the pattern
\begin{verbatim}
[thread model]-[toolchain]-[build type]
\end{verbatim}
The \verb+[thread model]+ option refers to the thread parallelism
model that the code will use by means of the \kokkos{} package in
\trilinos{}. Currently the sopported models are: \verb+serial+ that
works for all supported compilers, \verb+openmp+ that works with the
\verb+GCC+ and \verb+Intel+ toolchains, \verb+pthreads+ that works for
all supported compilers, and \verb+cuda+ that is only supported for
the \verb+GCC+ toolchain. The installation and configuration of the
Cuda framework is complex. Much more detailed information can be found
at \verb+http://developer.nvidia.com/cuda+.

Currently the options for \verb+[toolchain]+ are \verb+gcc+,
\verb+clang+ and \verb+intel+ if the Intel compilers are installed,
and for \verb+[build type]+ are \verb+debug+ (includes symbolic
information), \verb+release+ (optimization enabled), \verb+profile+
(symbolic information and optimization enabled for profiling) and
\verb+small+ (minimizes size of executables). The \verb+clang+
toolchain requires installation of the \verb+clang+ and
\verb+clang-devel+ packages.

Build directories are created within the \lcm{} top-level directory
and named according to the loaded module and package specified to the
build.sh script, e.g.:
\begin{verbatim}
albany-build-gcc-release
\end{verbatim}
In addition, for \trilinos{} an install directory similarly named is
created at the \lcm{} top-level directory.

\section{Configuring and compiling}

Assuming that we want to compile with a \verb+serial+ thread model
using the \verb+gcc+ tool chain in \verb+debug+ mode, load the
appropriate module:
\begin{verbatim}
module load serial-gcc-debug
\end{verbatim}
Now first configure and compile \trilinos{}. Within the top-level
\verb+LCM+ directory type:
\begin{verbatim}
./config-build.sh trilinos [# processors]
\end{verbatim}
For example, if you want to build using 16 processors, type:
\begin{verbatim}
./config-build.sh trilinos 16
\end{verbatim}
Finally, repeat the procedure for \albany{}:
\begin{verbatim}
./config-build.sh albany [# processors]
\end{verbatim}
For example, if you want to build a version of the code using 16
processors, type:
\begin{verbatim}
./config-build.sh albany 16
\end{verbatim}

Note that to compile a version of \albany{} with a specific thread model,
toolchain and build type, the corresponding version of \trilinos{} must
exist.

\section{After Initial Setup}
The procedure described above configures and compiles the code. From
now on, configuration is no longer required so you can rebuild the
code after any modification by simply using the \verb+build.sh+
script. For example:
\begin{verbatim}
./build.sh albany 16
\end{verbatim}
There are times when it is necessary to reconfigure, for example when
adding or deleting files under the \verb+LCM/Albany/src/LCM+
directory. This is generally anounced in the commit notices.

Also, note that both \trilinos{} and \albany{} are heavily templetized
C++ codes. Building the debug version of \albany{} requires large
amounts of memory because of the huge size of the symbolic information
required for debugging. Thus, if the compiling procedure stalls, try
reducing the number of processors.

\section{Running and Debugging \lcm{}} 

After building \albany{}, you might want to run and/or debug the code.
Tools were built in \trilinos{} (decomp, epu, etc.) that are necessary
for parallel execution. The environment created by loading the
appropriate module sets the proper paths so that the executables that
correspond to the type of build are accessible.

\section{Commiting Changes and Code Style}
\albany{} is a simulation code for researchers by researchers. As
such, vibrant development of new and exciting capabilities is strongly
encouraged. For these reasons, don't be afraid to commit changes to
the master git repository. We only ask that you don't break
compilation or testing. So please make sure that the tests pass before
you commit changes.

In addition, within \lcm{} we strongly encourage you to follow the C++
Google style guide that can be found at
\verb+http://google-styleguide.googlecode.com/svn/trunk/cppguide.html+.
This style is somewhat different to what is currently used in the rest
of \albany{}, but we believe that the Google style is better in that
it advocates more style differentiation between the different
syntactic elements of C++. This in turn makes reading code easier and
helps to avoid coding errors.

\end{document}
