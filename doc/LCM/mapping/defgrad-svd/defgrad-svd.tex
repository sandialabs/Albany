\documentclass{article}

\usepackage{algpseudocode}
\usepackage{alltt}
\usepackage{amsbsy}
\usepackage{amscd}
\usepackage{amsfonts}
\usepackage{amsmath}
\usepackage{amssymb}
\usepackage{amsthm}
\usepackage{boldtensors}
\usepackage{booktabs}
\usepackage{color}
\usepackage{enumitem}
\usepackage{fancyhdr}
\usepackage{float}
\usepackage{geometry}
\usepackage{graphicx}
\usepackage{microtype}
\usepackage{multirow}
\usepackage{siunitx}
\usepackage{supertabular}
\usepackage{times}
% Has to go before tikz
\usepackage[dvipsnames]{xcolor}
\usepackage{tikz}
\usepackage{tikz-3dplot}
\usepackage{url}
\usepackage{verbatim}
\usepackage{xspace}

\usepackage[nomessages]{fp}
\usepackage[english]{babel}
\usepackage[boxed]{algorithm}
\usepackage[FIGBOTCAP,TABTOPCAP,bf,tight]{subfigure}
\usepackage[T1]{fontenc}
\usepackage{ae,aecompl}
\usepackage[abs]{overpic}
\usepackage[normalem]{ulem}
\usepackage[font=footnotesize]{caption}

% cleveref must appear after hyperref to avoid conflicts
\usepackage[colorlinks]{hyperref}
\usepackage{cleveref}

\usepackage[backend=bibtex,maxcitenames=2]{biblatex}
\addbibresource{defgrad-svd.bib}

\usetikzlibrary{arrows,calc,decorations.markings,math,arrows.meta}
\usetikzlibrary{shapes.geometric,patterns}

%%%%%%%%%%%%%%%%%%%%%%%%%%%%%%%%%%%%%%%%%%%%%%%%%%%%%%%%%%%%%%%%%%%%%%%%
% editing
%%%%%%%%%%%%%%%%%%%%%%%%%%%%%%%%%%%%%%%%%%%%%%%%%%%%%%%%%%%%%%%%%%%%%%%%
%\usepackage[usenames,dvipsnames]{xcolor}
%\newcommand{\comment}[1]{\textcolor{blue}{[ \sc{#1} ]}} % comments
%\newcommand{\revise}[1]{\textcolor{red}{{#1}}} % revisions

%%%%%%%%%%%%%%%%%%%%%%%%%%%%%%%%%%%%%%%%%%%%%%%%%%%%%%%%%%%%%%%%%%%%%%%%
% abbreviations
%%%%%%%%%%%%%%%%%%%%%%%%%%%%%%%%%%%%%%%%%%%%%%%%%%%%%%%%%%%%%%%%%%%%%%%%
%\newcommand{\aref}[1]{{Appendix~\ref{#1}}}
%\newcommand{\eref}[1]{{(\ref{#1})}}
%\newcommand{\cref}[1]{{Ref.~\cite{#1}}}
%\newcommand{\sref}[1]{{Section~\ref{#1}}}
%\newcommand{\fref}[1]{{Fig.~\ref{#1}}}
%\newcommand{\tref}[1]{{Table~\ref{#1}}}
%\newcommand{\gref}[1]{{Algorithm~\ref{#1}}}
\newcommand{\cf}{{cf.}}
\newcommand{\ie}{{i.e.}}
\newcommand{\vs}{{vs.}}
\newcommand{\eg}{{e.g.}}
\newcommand{\etal}{{et al.}}

%%%%%%%%%%%%%%%%%%%%%%%%%%%%%%%%%%%%%%%%%%%%%%%%%%%%%%%%%%%%%%%%%%%%%%%%
% tensors
%%%%%%%%%%%%%%%%%%%%%%%%%%%%%%%%%%%%%%%%%%%%%%%%%%%%%%%%%%%%%%%%%%%%%%%%
\renewcommand{\vec}[1]{{\boldsymbol{#1}}}
\newcommand{\chib}{\vec{\chi}}

%% suffix
\newcommand{\ab}{\vec{a}}
%\newcommand{\bb}{\vec{b}}
\newcommand{\vb}{\vec{v}}
\newcommand{\Ab}{\vec{A}}
\newcommand{\Bb}{\vec{B}}
\newcommand{\Tb}{\vec{T}}

%% prefix
\newcommand{\bzero}{\vec{0}}
\newcommand{\bone}{\vec{1}}

\newcommand{\bA}{\vec{A}}
\newcommand{\bB}{\vec{B}}
\newcommand{\bC}{\vec{C}}
\newcommand{\bD}{\vec{D}}
\newcommand{\bE}{\vec{E}}
\newcommand{\bF}{\vec{F}}
\newcommand{\bG}{\vec{G}}
\newcommand{\bH}{\vec{H}}
\newcommand{\bI}{\vec{I}}
\newcommand{\bJ}{\vec{J}}
\newcommand{\bK}{\vec{K}}
\newcommand{\bL}{\vec{L}}
\newcommand{\bM}{\vec{M}}
\newcommand{\bN}{\vec{N}}
\newcommand{\bO}{\vec{O}}
\newcommand{\bP}{\vec{P}}
\newcommand{\bQ}{\vec{Q}}
\newcommand{\bR}{\vec{R}}
\newcommand{\bS}{\vec{S}}
\newcommand{\bT}{\vec{T}}
\newcommand{\bU}{\vec{U}}
\newcommand{\bV}{\vec{V}}
\newcommand{\bW}{\vec{W}}
\newcommand{\bX}{\vec{X}}
\newcommand{\bY}{\vec{Y}}
\newcommand{\bZ}{\vec{Z}}

\newcommand{\ba}{\vec{a}}
\newcommand{\bb}{\vec{b}}
\newcommand{\bc}{\vec{c}}
\newcommand{\bd}{\vec{d}}
\newcommand{\be}{\vec{e}}
\renewcommand{\bf}{\vec{f}}
\newcommand{\bg}{\vec{g}}
\newcommand{\bh}{\vec{h}}
\newcommand{\bi}{\vec{i}}
\newcommand{\bj}{\vec{j}}
\newcommand{\bk}{\vec{k}}
\newcommand{\bl}{\vec{l}}
\newcommand{\bm}{\vec{m}}
\newcommand{\bn}{\vec{n}}
\newcommand{\bo}{\vec{o}}
\newcommand{\bp}{\vec{p}}
\newcommand{\bq}{\vec{q}}
\newcommand{\br}{\vec{r}}
\newcommand{\bs}{\vec{s}}
\newcommand{\bt}{\vec{t}}
\newcommand{\bu}{\vec{u}}
\newcommand{\bv}{\vec{v}}
\newcommand{\bw}{\vec{w}}
\newcommand{\bx}{\vec{x}}
\newcommand{\by}{\vec{y}}
\newcommand{\bz}{\vec{z}}

\newcommand{\balpha}{\vec{\alpha}}
\newcommand{\bbeta}{\vec{\beta}}
\newcommand{\bgamma}{\vec{\gamma}}
\newcommand{\bdelta}{\vec{\delta}}
\newcommand{\bepsilon}{\vec{\epsilon}}
\newcommand{\bvarepsilon}{\vec{\varepsilon}}
\newcommand{\bzeta}{\vec{\zeta}}
\newcommand{\beeta}{\vec{\eta}}
\newcommand{\btheta}{\vec{\theta}}
\newcommand{\bvartheta}{\vec{\vartheta}}
\newcommand{\bkappa}{\vec{\kappa}}
\newcommand{\blambda}{\vec{\lambda}}
\newcommand{\bmu}{\vec{\mu}}
\newcommand{\bnu}{\vec{\nu}}
\newcommand{\bxi}{\vec{\xi}}
\newcommand{\bomicron}{\vec{o}}
\newcommand{\bpi}{\vec{\pi}}
\newcommand{\bvarpi}{\vec{\varpi}}
\newcommand{\brho}{\vec{\rho}}
\newcommand{\bvarrho}{\vec{\varrho}}
\newcommand{\bsigma}{\vec{\sigma}}
\newcommand{\bvarsigma}{\vec{\varsigma}}
\newcommand{\btau}{\vec{\tau}}
\newcommand{\bupsilon}{\vec{\upsilon}}
\newcommand{\bphi}{\vec{\phi}}
\newcommand{\bvarphi}{\vec{\varphi}}
\newcommand{\bchi}{\vec{\chi}}
\newcommand{\bpsi}{\vec{\psi}}
\newcommand{\bomega}{\vec{\omega}}

\newcommand{\bGamma}{\vec{\Gamma}}
\newcommand{\bDelta}{\vec{\Delta}}
\newcommand{\bTheta}{\vec{\Theta}}
\newcommand{\bLambda}{\vec{\Lambda}}
\newcommand{\bXi}{\vec{\Xi}}
\newcommand{\bPi}{\vec{\Pi}}
\newcommand{\bSigma}{\vec{\Sigma}}
\newcommand{\bUpsilon}{\vec{\Upsilon}}
\newcommand{\bPhi}{\vec{\Phi}}
\newcommand{\bPsi}{\vec{\Psi}}
\newcommand{\bOmega}{\vec{\Omega}}

%%%%%%%%%%%%%%%%%%%%%%%%%%%%%%%%%%%%%%%%%%%%%%%%%%%%%%%%%%%%%%%%%%%%%%%%
% matrix
%%%%%%%%%%%%%%%%%%%%%%%%%%%%%%%%%%%%%%%%%%%%%%%%%%%%%%%%%%%%%%%%%%%%%%%%
\newcommand{\mat}[1]{\mathsf{#1}}
\newcommand{\transpose}[1]{{#1}^{\text{T}}}
\newcommand{\conjtransla}[1]{{#1}^{\text{*}}}
\newcommand{\conjtransqm}[1]{{#1}^{\dagger}}

%% suffix
\newcommand{\ws}{\mat{w}}
\newcommand{\As}{\mat{A}}
\newcommand{\Ms}{\mat{M}}

%% prefix
\newcommand{\szero}{\mat{0}}
\newcommand{\sone}{\mat{1}}

\newcommand{\sA}{\mat{A}}
\newcommand{\sB}{\mat{B}}
\newcommand{\sC}{\mat{C}}
\newcommand{\sD}{\mat{D}}
\newcommand{\sE}{\mat{E}}
\newcommand{\sF}{\mat{F}}
\newcommand{\sG}{\mat{G}}
\newcommand{\sH}{\mat{H}}
\newcommand{\sI}{\mat{I}}
\newcommand{\sJ}{\mat{J}}
\newcommand{\sK}{\mat{K}}
\newcommand{\sL}{\mat{L}}
\newcommand{\sM}{\mat{M}}
\newcommand{\sN}{\mat{N}}
\newcommand{\sO}{\mat{O}}
\newcommand{\sP}{\mat{P}}
\newcommand{\sQ}{\mat{Q}}
\newcommand{\sR}{\mat{R}}
\newcommand{\sS}{\mat{S}}
\newcommand{\sT}{\mat{T}}
\newcommand{\sU}{\mat{U}}
\newcommand{\sV}{\mat{V}}
\newcommand{\sW}{\mat{W}}
\newcommand{\sX}{\mat{X}}
\newcommand{\sY}{\mat{Y}}
\newcommand{\sZ}{\mat{Z}}

\newcommand{\sa}{\mat{a}}
\renewcommand{\sb}{\mat{b}}
\renewcommand{\sc}{\mat{c}}
\newcommand{\sd}{\mat{d}}
\newcommand{\se}{\mat{e}}
\renewcommand{\sf}{\mat{f}}
\newcommand{\sg}{\mat{g}}
\newcommand{\sh}{\mat{h}}
% Avoid conflict with siunitx
%\newcommand{\si}{\mat{i}}
\newcommand{\sj}{\mat{j}}
\newcommand{\sk}{\mat{k}}
\renewcommand{\sl}{\mat{l}}
\newcommand{\sm}{\mat{m}}
\newcommand{\sn}{\mat{n}}
\newcommand{\so}{\mat{o}}
\renewcommand{\sp}{\mat{p}}
\newcommand{\sq}{\mat{q}}
\newcommand{\sr}{\mat{r}}
\renewcommand{\ss}{\mat{s}}
\newcommand{\st}{\mat{t}}
\newcommand{\su}{\mat{u}}
\newcommand{\sv}{\mat{v}}
\newcommand{\sw}{\mat{w}}
\newcommand{\sx}{\mat{x}}
\newcommand{\sy}{\mat{y}}
\newcommand{\sz}{\mat{z}}

\newcommand{\salpha}{\mat{\alpha}}
\newcommand{\sbeta}{\mat{\seta}}
\newcommand{\sgamma}{\mat{\gamma}}
\newcommand{\sdelta}{\mat{\delta}}
\newcommand{\sepsilon}{\mat{\epsilon}}
\newcommand{\svarepsilon}{\mat{\varepsilon}}
\newcommand{\szeta}{\mat{\zeta}}
\newcommand{\seta}{\mat{\eta}}
\newcommand{\stheta}{\mat{\theta}}
\newcommand{\svartheta}{\mat{\vartheta}}
\newcommand{\skappa}{\mat{\kappa}}
\newcommand{\slambda}{\mat{\lambda}}
\newcommand{\smu}{\mat{\mu}}
\newcommand{\snu}{\mat{\nu}}
\newcommand{\sxi}{\mat{\xi}}
\newcommand{\somicron}{\mat{o}}
\newcommand{\spi}{\mat{\pi}}
\newcommand{\svarpi}{\mat{\varpi}}
\newcommand{\srho}{\mat{\rho}}
\newcommand{\svarrho}{\mat{\varrho}}
\newcommand{\ssigma}{\mat{\sigma}}
\newcommand{\svarsigma}{\mat{\varsigma}}
\newcommand{\stau}{\mat{\tau}}
\newcommand{\supsilon}{\mat{\upsilon}}
\newcommand{\sphi}{\mat{\phi}}
\newcommand{\svarphi}{\mat{\varphi}}
\newcommand{\schi}{\mat{\chi}}
\newcommand{\spsi}{\mat{\psi}}
\newcommand{\somega}{\mat{\omega}}

\newcommand{\sGamma}{\mat{\Gamma}}
\newcommand{\sDelta}{\mat{\Delta}}
\newcommand{\sTheta}{\mat{\Theta}}
\newcommand{\sLambda}{\mat{\Lambda}}
\newcommand{\sXi}{\mat{\Xi}}
\newcommand{\sPi}{\mat{\Pi}}
\newcommand{\sSigma}{\mat{\Sigma}}
\newcommand{\sUpsilon}{\mat{\Upsilon}}
\newcommand{\sPhi}{\mat{\Phi}}
\newcommand{\sPsi}{\mat{\Psi}}
\newcommand{\sOmega}{\mat{\Omega}}

%%%%%%%%%%%%%%%%%%%%%%%%%%%%%%%%%%%%%%%%%%%%%%%%%%%%%%%%%%%%%%%%%%%%%%%%
% fourth-order tensors, common number sets
% latin/roman symbols only for now
%%%%%%%%%%%%%%%%%%%%%%%%%%%%%%%%%%%%%%%%%%%%%%%%%%%%%%%%%%%%%%%%%%%%%%%%

%% suffix

%% prefix
\newcommand{\bbA}{\mathbb{A}}
\newcommand{\bbB}{\mathbb{B}}
\newcommand{\bbC}{\mathbb{C}}
\newcommand{\bbD}{\mathbb{D}}
\newcommand{\bbE}{\mathbb{E}}
\newcommand{\bbF}{\mathbb{F}}
\newcommand{\bbG}{\mathbb{G}}
\newcommand{\bbH}{\mathbb{H}}
\newcommand{\bbI}{\mathbb{I}}
\newcommand{\bbJ}{\mathbb{J}}
\newcommand{\bbK}{\mathbb{K}}
\newcommand{\bbL}{\mathbb{L}}
\newcommand{\bbM}{\mathbb{M}}
\newcommand{\bbN}{\mathbb{N}}
\newcommand{\bbO}{\mathbb{O}}
\newcommand{\bbP}{\mathbb{P}}
\newcommand{\bbQ}{\mathbb{Q}}
\newcommand{\bbR}{\mathbb{R}}
\newcommand{\bbS}{\mathbb{S}}
\newcommand{\bbT}{\mathbb{T}}
\newcommand{\bbU}{\mathbb{U}}
\newcommand{\bbV}{\mathbb{V}}
\newcommand{\bbW}{\mathbb{W}}
\newcommand{\bbX}{\mathbb{X}}
\newcommand{\bbY}{\mathbb{Y}}
\newcommand{\bbZ}{\mathbb{Z}}

\newcommand{\bba}{\mathbb{a}}
\newcommand{\bbb}{\mathbb{b}}
\newcommand{\bbc}{\mathbb{c}}
\newcommand{\bbd}{\mathbb{d}}
\newcommand{\bbe}{\mathbb{e}}
\newcommand{\fb}{\mathbb{f}}
\newcommand{\bbg}{\mathbb{g}}
\newcommand{\bbh}{\mathbb{h}}
\newcommand{\bbi}{\mathbb{i}}
\newcommand{\bbj}{\mathbb{j}}
\newcommand{\bbk}{\mathbb{k}}
\newcommand{\bbl}{\mathbb{l}}
\newcommand{\bbm}{\mathbb{m}}
\newcommand{\bbn}{\mathbb{n}}
\newcommand{\bbo}{\mathbb{o}}
\newcommand{\bbp}{\mathbb{p}}
\newcommand{\bbq}{\mathbb{q}}
\newcommand{\bbr}{\mathbb{r}}
\newcommand{\bbs}{\mathbb{s}}
\newcommand{\bbt}{\mathbb{t}}
\newcommand{\bbu}{\mathbb{u}}
\newcommand{\bbv}{\mathbb{v}}
\newcommand{\bbw}{\mathbb{w}}
\newcommand{\bbx}{\mathbb{x}}
\newcommand{\bby}{\mathbb{y}}
\newcommand{\bbz}{\mathbb{z}}

%%%%%%%%%%%%%%%%%%%%%%%%%%%%%%%%%%%%%%%%%%%%%%%%%%%%%%%%%%%%%%%%%%%%%%%%
% third order tensors
%%%%%%%%%%%%%%%%%%%%%%%%%%%%%%%%%%%%%%%%%%%%%%%%%%%%%%%%%%%%%%%%%%%%%%%%
\newcommand{\cb}[1]{\boldsymbol{\mathcal{#1}}}

%% prefix
\newcommand{\cbA}{\cb{A}}
\newcommand{\cbB}{\cb{B}}
\newcommand{\cbC}{\cb{C}}
\newcommand{\cbD}{\cb{D}}
\newcommand{\cbE}{\cb{E}}
\newcommand{\cbF}{\cb{F}}
\newcommand{\cbG}{\cb{G}}
\newcommand{\cbH}{\cb{H}}
\newcommand{\cbI}{\cb{I}}
\newcommand{\cbJ}{\cb{J}}
\newcommand{\cbK}{\cb{K}}
\newcommand{\cbL}{\cb{L}}
\newcommand{\cbM}{\cb{M}}
\newcommand{\cbN}{\cb{N}}
\newcommand{\cbO}{\cb{O}}
\newcommand{\cbP}{\cb{P}}
\newcommand{\cbQ}{\cb{Q}}
\newcommand{\cbR}{\cb{R}}
\newcommand{\cbS}{\cb{S}}
\newcommand{\cbT}{\cb{T}}
\newcommand{\cbU}{\cb{U}}
\newcommand{\cbV}{\cb{V}}
\newcommand{\cbW}{\cb{W}}
\newcommand{\cbX}{\cb{X}}
\newcommand{\cbY}{\cb{Y}}
\newcommand{\cbZ}{\cb{Z}}

%%%%%%%%%%%%%%%%%%%%%%%%%%%%%%%%%%%%%%%%%%%%%%%%%%%%%%%%%%%%%%%%%%%%%%%%
% sets
%%%%%%%%%%%%%%%%%%%%%%%%%%%%%%%%%%%%%%%%%%%%%%%%%%%%%%%%%%%%%%%%%%%%%%%%
\newcommand{\set}[1]{\mathcal{#1}}

%% suffix
\newcommand{\Gc}{\set{G}}

%% prefix
\newcommand{\cA}{\set{A}}
\newcommand{\cB}{\set{B}}
\newcommand{\cC}{\set{C}}
\newcommand{\cD}{\set{D}}
\newcommand{\cE}{\set{E}}
\newcommand{\cF}{\set{F}}
\newcommand{\cG}{\set{G}}
\newcommand{\cH}{\set{H}}
\newcommand{\cI}{\set{I}}
\newcommand{\cJ}{\set{J}}
\newcommand{\cK}{\set{K}}
\newcommand{\cL}{\set{L}}
\newcommand{\cM}{\set{M}}
\newcommand{\cN}{\set{N}}
\newcommand{\cO}{\set{O}}
\newcommand{\cP}{\set{P}}
\newcommand{\cQ}{\set{Q}}
\newcommand{\cR}{\set{R}}
\newcommand{\cS}{\set{S}}
\newcommand{\cT}{\set{T}}
\newcommand{\cU}{\set{U}}
\newcommand{\cV}{\set{V}}
\newcommand{\cW}{\set{W}}
\newcommand{\cX}{\set{X}}
\newcommand{\cY}{\set{Y}}
\newcommand{\cZ}{\set{Z}}

%%%%%%%%%%%%%%%%%%%%%%%%%%%%%%%%%%%%%%%%%%%%%%%%%%%%%%%%%%%%%%%%%%%%%%%%
% operators
%%%%%%%%%%%%%%%%%%%%%%%%%%%%%%%%%%%%%%%%%%%%%%%%%%%%%%%%%%%%%%%%%%%%%%%%
\newcommand{\op}[1]{\operatorname{\text{#1}}}
\newcommand{\cddot}{\op{:}}
\newcommand{\tr}{\op{tr}}
\renewcommand{\div}{\op{div}}
\newcommand{\grad}{\op{grad}}
\newcommand{\curl}{\op{curl}}
\newcommand{\Div}{\op{Div}}
\newcommand{\Grad}{\op{Grad}}
\newcommand{\Curl}{\op{Curl}}
\newcommand{\vol}{\op{vol}}
\newcommand{\dev}{\op{dev}}
\newcommand{\Vol}{\op{Vol}}
\newcommand{\Dev}{\op{Dev}}
\newcommand*\diff{\mathop{}\!\mathrm{d}}


\newcommand{\lx}[2]{{\color{blue}{#1}} {\color{blue}{#2}}}

\newcommand{\alexa}{\textsc{Alexa}}
\newcommand{\minitensor}{\textsc{MiniTensor}}
\newcommand{\svd}{\textsc{SVD}}
\newcommand{\trilinos}{\textsc{Trilinos}}

\newcommand{\diag}{\op{diag}}
\renewcommand{\skew}{\op{skew}}

\providecommand{\keywords}[1]{\textbf{Keywords:} #1}

\begin{document}

\pagestyle{fancyplain}

\lhead [\fancyplain{}{\emph{A. Mota}}]
{\fancyplain{}{\emph{A. Mota}}}

\rhead
[\fancyplain{}{\emph{Update and map deformation gradient via SVD}}]
{\fancyplain{}{\emph{Update and map deformation gradient via SVD}}}

\title{Update and map deformation gradient via SVD}

\author{\Large Alejandro Mota$^1$\thanks{Email: amota@sandia.gov}
  \\
  \\
  $^1$Mechanics of Materials\\
  Sandia National Laboratories\\
  Livermore, CA 94550, USA\\
}

\date{\today}

\maketitle

\begin{abstract}
  We advocate the use of the singular value decomposition (\svd{}) for
  the storage, update, and mapping of the deformation gradient for
  solid mechanics problems with extremely large deformations.
\end{abstract}

\keywords{mapping, internal variables, kinematics, deformation
  gradient, singular value decomposition}

\section{Introduction}
\label{sec:intro}

Our previous work on mapping and updating internal variables via $L_2$
projections and Lie group interpolations did not contemplate the case
of extremely large deformation \autocite{Mota.etal:2013}. When very
large deformation is involved, the deformation gradient $~F$ may have
a rotation component that has an angle of rotation
$\theta \approx n \pi$, where $n \in {"Z}_{\ge 0}$. In that situation,
algorithms to compute the principal logarithm of $~F$ may fail or
require the use of complex arithmetic. This is undesirable. On the
other hand, \textcite{Mota.etal:2013} advocate the use of the polar
decomposition of the deformation gradient $~F = ~R ~U$, compute the
logarithm of the rotation $~R$ and stretch $~U$ separately, map, and
then reconstitute the deformation gradient field by means of the
exponential map. This method requires the use of five tensor
algorithms: polar decomposition; logarithmic map for rotations;
logarithmic map for symmetric positive definite tensors; exponential
map for symmetrc tensors; and the exponential map for rotations. The
use of specialized algorithms for rotations and symmetric tensors
seeks to exploit their structure to minimize floating point
operations.

Herein, we advocate the use of the singular value decomposition
(\svd{}) to adress the case of very large deformation. An additional
benefit is the reduction of storage and decrease of the number of
tensor algorithms from five to three, as follows. First compute the
\svd{} of the deformation gradient $~F = ~U ~S ~V^{\text{T}}$, then
compute the logarithmic map of each of the components, map, apply the
exponential map, and reconstitute the mapped deformation
gradient. Since the logarithm of each of the components of the \svd{}
requires three scalars, the amount of storage is the same as for the
deformation gradient $~F$ or its principal logarithm $\log
~F$. Furthermore, the required logarithmic and exponential maps are
those corresponding to either scalars or proper
rotations. Additionally, provided that the deformation gradient is
non-singular, the singular values are guaranteed to be positive. This
is all described in detail in \Cref{alg:defgrad-svd}.

\begin{algorithm}[htbp]
  \footnotesize
  \begin{algorithmic}[1]

    \State
    $~u \gets ~0, ~s \gets ~0, ~v \gets ~0$
    \Comment
    Rotation pseudovectors of initial $~U=~I, ~V=~I$,
    Lie algebra vector of initial $~S=~I$

    \State $t \gets 0$

    \Repeat
    \Comment{Time integration loop}

    \State \label{alg:skew-diag}
    $~U \gets \exp \skew ~u, \enspace ~S \gets \diag \exp ~s, \enspace
    ~V \gets \exp \skew ~v$

    \State
    $~F \gets ~U ~S ~V^\text{T}$
    
    \State
    Compute increment of deformation gradient $\triangle ~F$

    \State \label{alg:svd}
    $\left(\triangle ~U, \triangle ~S, \triangle ~V\right)
    \gets \svd(\triangle ~F)$

    \State 
    $~U \gets \triangle ~U \cdot ~U,
    \enspace
    ~S \gets \triangle ~S \cdot ~S,
    \enspace
    ~V \gets \triangle ~V \cdot ~V$
    
    \State 
    $~u \gets \skew \log ~U,
    \enspace
    ~s \gets \log \diag ~S,
    \enspace
    ~v \gets \skew \log ~V$
    
    \If{mapping required}
    
    \State
    Map $~u, ~s, ~v$ as vectors, using additive (Lie algebra) interpolation.
    
    \EndIf

    \State
    Compute time step $\triangle t$

    \State $t \gets t + \triangle t$

    \Until{$t \ge T_{\text{final}}$}

  \end{algorithmic}
  \caption{Mapping algorithm for the deformation gradient $~F$ via the
    \svd{}.}
  \label{alg:defgrad-svd}
\end{algorithm}

The crucial assumption in this algorithm is that the angles of
rotation of $\triangle ~U$ and $\triangle ~V$ in \Cref{alg:svd},
corresponding to the \svd{} of the incremental deformation gradient
$\triangle ~F$, are less than $\pi$.

\section{Tensor operators}
\label{sec:tensor-operators}

The steps outlined in \Cref{alg:defgrad-svd} require the computation
of several tensor operators. The skew symmetric operator in
\Cref{alg:skew-diag} is defined in such a way that if applied to a
vector $~v \in {"R}^3$, it returns a skew symmetric tensor
$\check{~v} \in so(3)$, and if applied to a skew symmetric tensor
$\check{~v} \in so(3)$, it returns a vector $~v \in {"R}^3$. In both
cases, the operator is constructed such that
$\check{\bv} \cdot \bu \equiv \bv \times \bu \ \forall \ \bu \in
{"R}^3$, that is
\begin{equation} \label{eq:skew}
  ~v = \skew \check{~v}
  \quad
  \Leftrightarrow
  \quad
  \check{~v} = \skew ~v
  \quad
  \Leftrightarrow
  \quad
  ~v =
  \begin{pmatrix}
    v_0 \\ v_1 \\ v_2
  \end{pmatrix}
  \quad
  \Leftrightarrow
  \quad
  \check{~v} =
  \begin{pmatrix}
    0 & -v_2 & v_1
    \\
    v_2 & 0 & -v_0
    \\
    -v_1 & v_0 & 0
  \end{pmatrix}.
\end{equation}
The diagonal operator also in \Cref{alg:skew-diag} is defined in a
similar fashion as
\begin{equation} \label{eq:diag}
  ~v = \diag \hat{~v}
  \quad
  \Leftrightarrow
  \quad
  \hat{~v} = \diag ~v
  \quad
  \Leftrightarrow
  \quad
  ~v =
  \begin{pmatrix}
    v_0 \\ v_1 \\ v_2
  \end{pmatrix}
  \quad
  \Leftrightarrow
  \quad
  \check{~v} =
  \begin{pmatrix}
    v_0 & 0 & 0
    \\
    0 & v_1 & 0
    \\
    0 & 0 & v_2
  \end{pmatrix}.
\end{equation}
The exponential and logarithmic maps for a vector corresponding to a
diagonal tensor are trivial; they only require the logarithm and
exponential of each of the vector entries as scalars. For the
logarithmic map of the rotation, there exist explicit formulas
\autocite{Park.Ravani:1997}. The angle of rotation is given by
\begin{equation}
  \theta := \cos^{-1} \left[  \tfrac{1}{2} (\tr ~R - 1) \right],
  \quad
  ~R \in SO(3),
  \quad
  \theta \in [0, \pi],
\end{equation}
and the logarithm of the rotation is determined by
\begin{equation} \label{eq:log-rotation}
  \begin{split}
    \begin{aligned}
      \log ~R =
      \begin{cases}
        ~0 \in so(3),
        &
        \text{ if }\theta = 0;
        \\[0.4em]
        \displaystyle{
          \frac{\theta}{2 \sin \theta} (~R - ~R^\text{T}) \in so(3),
        }
        &
        \text{ if }\theta \in (0, \pi);
        \\[0.7em]
        \pm \pi \check{~w} \in so(3),
        &
        \text{ if }\theta = \pi;
      \end{cases}
    \end{aligned}
  \end{split}
\end{equation}
in which $\check{~w} \in so(3)$ is the skew-symmetric tensor such that
$~w = \skew \check{~w}$ is the eigenvector corresponding to the
eigenvalue of 1 of $~R$, and the sign is selected according to
continuity conditions from the field in the
neighborhood. Nevertheless, since the assumption is that the angle of
rotation $\theta \in [0, \pi)$, the case $\theta = \pi$ is best
handled as an exception. For the exponential map of the rotation, define
\begin{equation} \label{eq:angle-as-norm-skew}
  \theta := ||~w||
\end{equation}
as the angle of rotation for the rotation pseudovector $~w$. Then the
exponential map for the skew-symmetric tensor $\check{~w} := \skew ~w$
is given by the expression
\begin{equation} \label{eq:Rodrigues-formula}
  \begin{split}
    \begin{aligned}
      \exp \check{~w} =
      \begin{cases}
        ~I \in SO(3),
        &
        \text{ if }\theta = 0;
        \\
        \displaystyle{
          ~I + \frac{\sin \theta}{\theta} \check{~w} +
          \frac{(1 - \cos \theta)}{\theta^2} \check{~w}^2 \in SO(3),
        }
        &
        \text{ if }\theta > 0;
      \end{cases}
    \end{aligned}
  \end{split}
\end{equation}
which is often accredited to Rodrigues \autocite{Gallier:2011}. This
completes the description of the algorithm.

\printbibliography

\end{document}
