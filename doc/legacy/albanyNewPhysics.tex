\title{Putting New Physics into \textit{Albany}}


\author{
Irina K. Tezaur}



\date{\today}

\documentclass[11pt]{article}
\usepackage{amsmath}
\usepackage{amssymb}
\usepackage{amsthm}
\usepackage{array}
\usepackage{xy}
\usepackage{epsfig}
\usepackage{fancyhdr}
\usepackage{setspace}
\usepackage{graphicx}
\usepackage{wrapfig}
\usepackage{floatflt}
\usepackage{mathrsfs}
\usepackage{url}
\usepackage{subfigure}
\usepackage{lscape}
\usepackage{algorithm,algorithmic}
\usepackage{multirow}
\usepackage{bm}
\usepackage{color}

\oddsidemargin  -0.3in \evensidemargin 0.0in \textwidth      7in
\textheight     8.5in \headheight     -0.3in \topmargin      0.0in


\newcommand{\x}{\mathbf{x}}
\newcommand{\Bphi}{{\bm \phi}}
\newcommand{\dg}{$^\circ$ }
\newcommand{\Pe}{\text{Pe} }
\newcommand{\Peclet}{\textrm{P\'{e}clet}}
\newcommand{\uvec}{\mathbf{u}}
\newcommand{\uhatvec}{\hat{\mathbf{u}}}
\newcommand{\bM}{\mathbf{M}}
\newcommand{\bLambda}{\bm{\Lambda}}
\newcommand{\bQ}{\mathbf{Q}}
\newcommand{\btildeM}{\tilde{\mathbf{M}}}
\newcommand{\bD}{\mathbf{D}}
\newcommand{\bK}{\mathbf{K}}
\newcommand{\bA}{\mathbf{A}}
\newcommand{\bU}{\mathbf{U}}
\newcommand{\bF}{\mathbf{F}}
\newcommand{\bB}{\mathbf{B}}
\newcommand{\bC}{\mathbf{C}}
\newcommand{\bX}{\mathbf{X}}
\newcommand{\bI}{\mathbf{I}}
\newcommand{\bT}{\mathbf{T}}
\newcommand{\bE}{\mathbf{E}}
\newcommand{\bL}{\mathbf{L}}
\newcommand{\bR}{\mathbf{R}}
\newcommand{\zero}{\mathbf{0}}
\newcommand{\avec}{\mathbf{a}}
\newcommand{\be}{\mathbf{e}}
\newcommand{\xvec}{\mathbf{x}}
\newcommand{\bn}{\mathbf{n}}
\newcommand{\vvec}{\mathbf{v}}
\newcommand{\zvec}{\mathbf{z}}
\newcommand{\fvec}{\mathbf{f}}
\newcommand{\gvec}{\mathbf{g}}
\newcommand{\bq}{\mathbf{q}}
\newcommand{\bV}{\mathbf{V}}
\newcommand{\utilde}{\tilde{\mathbf{u}}}
\newcommand{\vtilde}{\tilde{\mathbf{v}}}
\newcommand{\bmu}{{\bm \mu}}
\newcommand{\bpsi}{{\bm \psi}}
\newcommand{\bphi}{{\bm \phi}}
\newcommand{\bomega}{{\bm \omega}}
\newcommand{\bxi}{{\bm \xi}}
\newcommand{\bzeta}{{\bm \zeta}}
\newcommand{\bOmega}{{\bm \Omega}}
\newcommand{\bGamma}{{\bm \Gamma}}
\newcommand{\Vspace}{\mathcal{\bm V}}
\newcommand{\Wspace}{\mathcal{\bm W}}
\newcommand{\Qspace}{\mathcal{Q}}
\newcommand{\Pspace}{\mathcal{P}}

\begin{document}


\maketitle

This document describes the key steps to putting in a new problem with a new set of physics (PDEs) into 
the \textit{Albany} code base. 

\begin{enumerate}
\item Obtain and build TPLs required for \textit{Trilinos} (HDF5, Netcdf, Boost), \textit{Trilinos} and \textit{Albany} (see instructions 
on the \textit{Albany} WiKi: \url{https://github.com/gahansen/Albany/wiki/Building-Albany-and-supporting-tools}, \url{https://github.com/gahansen/Albany/wiki/Building-the-new-Albany}).
\item Find a problem in \textit{Albany} that is similar to your problem (e.g., similar PDEs, same \# dofs/node, etc.). Lets say the problem you find is has the name ``Original Problem" with evaluators {\tt problems/OrigProblem.hpp}, {\tt problems/OrigProblem.cpp}, {\tt evaluators/OrigProblemResid.hpp}, {\tt evaluators/OrigProblemResid}$\_${\tt Def.hpp}, {\tt evaluators/OrigProblemResid.cpp}.   
\item Lets say you want to create a problem called ``New Problem".  First, copy the following: 
\begin{itemize}
\item {\tt cd Albany/src/problems} \\
 {\tt cp OrigProblem.hpp NewProblem.hpp}\\
 {\tt cp OrigProblem.cpp NewProblem.cpp}.
\item {\tt cd Albany/src/evaluators}\\ 
 {\tt cp OrigProblemResid.hpp NewProblemResid.hpp}\\
 {\tt cp OrigProblemResid.cpp NewProblemResid.cpp}\\
 {\tt cp OrigProblemResid}$\_${\tt Def.hpp NewProblemResid}$\_${\tt Def.hpp}. 
\item {\tt cd Albany/examples} \\ 
{\tt mkdir NewProblem}\\
{\tt cd NewProblem}\\
{\tt cp ../OldProblem/CMakeLists.txt .}\\
{\tt cp ../OldProblem/input*.xml .}  
\end{itemize} 
\item Rename the classes in the new files under {\tt Albany/src/problems} and {\tt Albany/src/evaluators}. 
\item Edit {\tt CMakeLists.txt} and {\tt examples/NewProblem} 
to add new files and directories that have just been created.
\item  Edit {\tt Albany/src/problems/Albany}$\_${\tt ProblemFactory.cpp} to add your new method and a constructor for it, e.g.,\\ 
{\tt else if (method == "New Problem") }\\
{\tt    strategy = rcp(new Albany::NewProblem(problemParams, paramLib));}
\item Edit {\tt Albany/src/examples/NewProblem/input*.xml} files, changing ``Problem" name to ``New Problem". 
\item Clean up NewProblem files in {\tt Albany/src/evaluators} and {\tt Albany/src/problems} -- remove extraneous stuff, change \# dofs per node, if necessary.  You may need to add dependencies, but this can be done at a later time.  
\item Put your new PDEs (weak for of the residual) in {\tt NewProblemResid}$\_${\tt Def.hpp}.  
\item You may want to write additional evaluators for some of the terms in the PDEs, e.g., viscosity, source term, etc.
\item Edit {\tt Albany/src/examples/NewProblem/input*.xml} files to specify your desired initial conditions, boundary conditions and mesh 
(``Dirichlet BCs", ``Neumann BCs", ``Discretization" sections).  Also specify any parameters, if applicable.  Change the solver, if applicable/desired.  
\end{enumerate}


\end{document}
