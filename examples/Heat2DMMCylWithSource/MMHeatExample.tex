\documentclass[12pt]{article}
\usepackage{fullpage}
\usepackage{graphicx}
\usepackage{color}
\usepackage{amsmath}
\usepackage{amsfonts}
\usepackage{listings}
\usepackage{framed}
\usepackage[pdftex,pdfborder={0 0 0}]{hyperref}

\newcommand{\uo}{\mathrm{UO}_2}

%opening
\title{2D Multimaterial Heat Conduction in Cylindrical Geometry With Source}
\author{Glen Hansen}

\begin{document}
\maketitle

\begin{figure}[!htbp]
\includegraphics[width=6.5in,trim=100 200 200 100,clip]{cyl}
\caption{Concentric cylindrical model with multiple regions of different thermal conductivity. Region 1 has
conductivity $k_1$, radius $r_1$, internal heat generation term $\dot{q}$, and is made of $\uo$.}
\label{fig:domain}
\end{figure}


This document describes the Albany Heat2DMMCylWithSource test problem. It is a
simple heat conduction problem, with a constant volumetric source in Region~1
(shown in Fig.~\ref{fig:domain}), that is surrounded by two different materials
to form concentric cylinders. Region~1 is meant to be a reactor fuel that
generates heat by radioactive decay at a constant rate $\dot{q}$. A ceramic
material of moderate thermal conductivity $k_1$ and made of $\uo$ fills the
region. The radius is $r_1$, and the outside temperature of this area is $T_1$.

Zircalloy metal, the fuel cladding, surrounds this fuel. The cladding has an outside radius
$r_2$, a conductivity $k_2$, and an outside temperature $T_2$. Lastly, air (or vacuum) surrounds the
cladding. The air region has a thermal conductivity $k_3$, radius $r_3$, and outside temperature $T_3$.
The idea behind this test is to calculate the effective thermal conductivity $k_3$ of the air/vacuum environment
that surrounds the fuel/cladding while it is ``dryed.'' Given a decay heat $\dot{q}$, what is the thermal
conductivity of this material needed such that the peak internal temperature of the fuel (Region~1) does not
exceed $400 C$ given a surrounding environment temperature $T_3$ of $40 C$.

This test is meant to provide a simple steady-state validation of initial capabilities needed to calculate
such things. Here, we model the concentric cylinders in two stages. The first stage uses a simplified form
of (3.28) in
\cite{incropera81}:
%
\begin{equation}
q_r = \frac{T_{\infty,1} - T_{\infty, 4}}{\frac{1}{2\pi r_1 L h_1} + \frac{\ln (r_2 / r_1)}{2\pi k_A L} +
\frac{\ln (r_3 / r_2)}{2 \pi k_B L} + \frac{\ln(r_4/r_3)}{2\pi k_C L} + \frac{1}{2\pi r_4 L h_4}}
\end{equation}

For the problem described in Fig.~\ref{fig:domain}, and assuming that the length of the problem $L$ is
unity, this becomes
%
\begin{equation}
q_r = \frac{T_{1} - T_{3}}{\frac{\ln (r_2 / r_1)}{2\pi k_2} +
\frac{\ln (r_3 / r_2)}{2 \pi k_3} }
\label{eq:resist}
\end{equation}

Now, note that an overall energy balance at a radius of $r_1$ will give
%
\begin{equation}
q_r(r=r_1, \mathrm{inside}) =
q_r(r=r_1, \mathrm{outside}) 
\end{equation}
or
\begin{equation}
\dot{q} (\pi r_1^2) =
\frac{T_{1} - T_{3}}{\frac{\ln (r_2 / r_1)}{2\pi k_2} +
\frac{\ln (r_3 / r_2)}{2 \pi k_3} }
\end{equation}
or
\begin{equation}
\frac{\dot{q} r_1^2}{2} \left[
\frac{\ln (r_2 / r_1)}{k_2} +
\frac{\ln (r_3 / r_2)}{k_3} \right] =
T_{1} - T_{3}
\end{equation}
or simply
\begin{equation}
\frac{\dot{q} r_1^2}{2} \left[ \quad \cdot \quad \right] =
T_{1} - T_{3}
\label{eq:bal}
\end{equation}

Lastly, note that (3.52) in \cite{incropera81}
%
\begin{equation}
T(r) = \frac{\dot{q} r_o^2}{4 k} \left( 1 - \frac{r^2}{r_o^2}\right) + T_s
\end{equation}
%
may be rewritten for our case of the center cylinder $r_1$ as
%
\begin{equation}
T(r) = \frac{\dot{q}}{4 k_1} \left( r_1^2 - r^2\right) + T_1
\label{eq:fuel_t_1}
\end{equation}

As one is typically most interested in the temperature profile inside the fuel region~1, we can rewrite
\eqref{eq:bal} in terms of $T_1$
%
\begin{equation}
T_{1} =
\frac{\dot{q} r_1^2}{2} \left[ \quad \cdot \quad \right]
+ T_{3}
\end{equation}
%
and substitute this into \eqref{eq:fuel_t_1} to get
%
\begin{equation}
T(r) = \frac{\dot{q}}{4 k_1} \left( r_1^2 - r^2\right) + 
\frac{\dot{q} r_1^2}{2} \left[
\frac{\ln (r_2 / r_1)}{k_2} +
\frac{\ln (r_3 / r_2)}{k_3} \right] + T_3
\label{eq:fuel_t_2}
\end{equation}

\section{Use Case}

Here, we will vary $r$ in \eqref{eq:fuel_t_2} to see the temperature variation in region 1, the $\uo$ fuel area. 
The test case for this code is to compare long-time transient results in this region to this steady derivation.


\begin{thebibliography}{10}

\bibitem{incropera81}
Frank P. Incropera and David P. DeWitt, Fundamentals of Heat Transfer,
John Wiley \& Sons, New York, NY, ISBN 0-471-42711-X, 1981.

\end{thebibliography}

\end{document}
